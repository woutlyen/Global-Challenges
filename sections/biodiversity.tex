\documentclass[../summary.tex]{subfiles}

\begin{document}
	
\section{Biodiversity}
\subsection{What is biodiversity and where does it come from?}
\subsubsection{What is biodiversity?}

Biodiversity is a term that refers to the variety of life forms on Earth. Coined in 1985 by biologist Walter Rosen during the National Forum on Biodiversity. The term gained widespread use without a precise definition until the UN Convention on Biological Diversity in Rio de Janeiro, which provided a clear definition. \\
\\
Biodiversity was defined as "the variability among living organisms from all sources, including terrestrial, marine, and other aquatic ecosystems, and the ecological complexes of which they are parts."\\
\\
Biodiversity encompasses diversity within species, between species, and of ecosystems. It is measured at three levels: genetic diversity, species diversity, and ecosystem diversity. Genetic diversity refers to the variation among individuals within the same species. Species diversity is the number of known species at a particular location, with approximately 2.2 million species known in 2022. Ecosystem diversity encompasses a wide range of ecosystems, from forests to aquatic systems, contributing to the overall diversity on Earth.\\
\\
In 2011, it was estimated that in total, 8.8 million species live on the earth. This means that we still don't know 75\% of all living species today.

\subsubsection{Where does biodiversity come from?}

The history of life on Earth, with an estimated 8.7 million eukaryote species, is a story that spans billions of years. Understanding when and how these species appeared is made possible through the study of fossils, which provides insights into the emergence and extinction of various life forms. Key milestones include:

\begin{itemize}
	\item \textbf{Emergence of Life}: The Earth is approximately 4.5 billion years old. Bacterial life began around 3.7 billion years ago.
	\item \textbf{Cambrian Explosion}: Roughly 570 million years ago, the Cambrian explosion saw a rapid increase in species diversity, possibly due to rising oxygen levels.
	\item \textbf{Emergence of Life Forms}: Fish appeared 500 million years ago, land plants around 470 million years ago, and mammals roughly 200 million years ago.
	\item \textbf{Speciation}: The process behind the increasing species richness is speciation, which involves different populations of the same species evolving into distinct species, driven by natural selection.
	\item \textbf{Current Species}: Today's species represent only a fraction of all that have existed (between 2\% and 5\%).
	\item \textbf{Extinctions}: Extinctions have occurred throughout history, both gradually (background extinctions) and catastrophically (mass extinctions).
	\item \textbf{Mass Extinctions}: Five major mass extinctions, caused by events like meteorite impacts and volcanic activity, have occurred since the Cambrian. The most famous was the dinosaur-extincting event 65 million years ago, linked to a meteorite impact. The most severe was the Permian-Triassic extinction, where over 90\% of species disappeared.
\end{itemize}
\ \\
Recovery from mass extinctions has taken at least 10 million of years, underscoring the importance of preserving the diverse life on our planet.
\newpage

\subsection{Biodiversity loss and humans}
\subsubsection{Historical extinction}



\end{document}