\documentclass[../summary.tex]{subfiles}

\begin{document}
	
\section{Biodiversity}
\subsection{What is biodiversity and where does it come from?}
\subsubsection{What is biodiversity?}

Biodiversity is a term that refers to the variety of life forms on Earth. Coined in 1985 by biologist Walter Rosen during the National Forum on Biodiversity. The term gained widespread use without a precise definition until the UN Convention on Biological Diversity in Rio de Janeiro, which provided a clear definition. \\
\\
Biodiversity was defined as "the variability among living organisms from all sources, including terrestrial, marine, and other aquatic ecosystems, and the ecological complexes of which they are parts."\\
\\
Biodiversity encompasses diversity within species, between species, and of ecosystems. It is measured at three levels: genetic diversity, species diversity, and ecosystem diversity. Genetic diversity refers to the variation among individuals within the same species. Species diversity is the number of known species at a particular location, with approximately 2.2 million species known in 2022. Ecosystem diversity encompasses a wide range of ecosystems, from forests to aquatic systems, contributing to the overall diversity on Earth.\\
\\
In 2011, it was estimated that in total, 8.8 million species live on the earth. This means that we still don't know 75\% of all living species today.

\subsubsection{Where does biodiversity come from?}

The history of life on Earth, with an estimated 8.7 million eukaryote species, is a story that spans billions of years. Understanding when and how these species appeared is made possible through the study of fossils, which provides insights into the emergence and extinction of various life forms. Key milestones include:

\begin{itemize}
	\item \textbf{Emergence of Life}: The Earth is approximately 4.5 billion years old. Bacterial life began around 3.7 billion years ago.
	\item \textbf{Cambrian Explosion}: Roughly 570 million years ago, the Cambrian explosion saw a rapid increase in species diversity, possibly due to rising oxygen levels.
	\item \textbf{Emergence of Life Forms}: Fish appeared 500 million years ago, land plants around 470 million years ago, and mammals roughly 200 million years ago.
	\item \textbf{Speciation}: The process behind the increasing species richness is speciation, which involves different populations of the same species evolving into distinct species, driven by natural selection.
	\item \textbf{Current Species}: Today's species represent only a fraction of all that have existed (between 2\% and 5\%).
	\item \textbf{Extinctions}: Extinctions have occurred throughout history, both gradually (background extinctions) and catastrophically (mass extinctions).
	\item \textbf{Mass Extinctions}: Five major mass extinctions, caused by events like meteorite impacts and volcanic activity, have occurred since the Cambrian. The most famous was the dinosaur-extinction event 65 million years ago, linked to a meteorite impact. The most severe was the Permian-Triassic extinction, where over 90\% of species disappeared.
\end{itemize}
\ \\
Recovery from mass extinctions has taken at least 10 million of years, underscoring the importance of preserving the diverse life on our planet.
\newpage

\subsection{Biodiversity loss and humans}
\subsubsection{Historical extinction}

The Late Quaternary or Pleistocene Megafaunal extinction wave, caused by human hunting, occurred around 60,000 years ago when Homo sapiens colonized the globe. This wave resulted in the extinction of large animals, such as the European lion and the giant sloths. The overkill hypothesis, developed by Paul Martin in the 1970s, suggests that human hunting caused the extinction of megafauna roaming the globe during the Pleistocene. \\
\\
Modern humans colonized the globe in four waves, with the last wave occurring less than 1000 years ago. The extinction of large mammals in Australia (almost 90\%) and Europe (35\%) coincided with the extinction of about 85\% of American megafauna. There is a striking parallel between timing of colonization by humans and extinctions. The effects of the extinction of one species can cascade throughout the entire food web.\\
\\
Comparing data on body size of animals that have gone extinct over 10,000 years ago reveals selective impact on larger animals, with small species escaping extinctions. Large mammals and top predators, like the elephant and lion, survived humans in Africa, possibly due to co-evolution and adaptation to hunting.

\subsubsection{Recent biodiversity loss}
\label{sec:recent-diversity-loss}
Humans had a strong hand in recent species extinctions. Since 1500, around 160 bird species, 60 mammal species and 100 species of other vertebrate groups, like reptiles and amphibians,  have been extinct. A famous example of an extinct species is the Dodo, as seen in figure \ref{fig:recentbiodiversityloss}.

\begin{figure}[H]
	\centering
	\includegraphics[width=0.6\linewidth]{../images/recent_biodiversity_loss}
	\caption{Cumulative number of extinct species since 1500}
	\label{fig:recentbiodiversityloss}
\end{figure}
\ \\

A lot of species are not extinct yet, but face strong declines in numbers.  

An example of this is the tiger, which occurred in large parts of Asia. But now, it has been pushed back to a small part of it's original range.  The global population declined by 50\% between 1998 and 2015.

\begin{figure}[H]
	\centering
	\includegraphics[width=0.9\linewidth]{../images/tiger_population}
	\caption{Tiger population in Asia}
	\label{fig:tigerpopulation}
\end{figure}
\ \\
The Living Planet Index (LPI) [figure \ref{fig:livingplanetindex}] shows the evolution of the average population size of 21 thousand animal populations around the globe. It shows a decline of 68\% since it's start in 1970. Keep in mind that the LPI is strongly influenced by a minority of populations with a very strong decrease in numbers [figure \ref{fig:lpi}].

\begin{figure}[H]
	\centering
	\includegraphics[width=0.8\linewidth]{../images/living_planet_index}
	\caption{The Living Planet Index}
	\label{fig:livingplanetindex}
\end{figure}

\begin{figure}[H]
	\centering
	\includegraphics[width=0.8\linewidth]{../images/LPI}
	\caption{Impact of most severe declines on the Living Planet Index}
	\label{fig:lpi}
\end{figure}
\ \\
The IUCN Red List is another important monitoring program which monitors the extinction risk of individual species. Based on some criteria, species are assigned to one of the six classes on the list. The tiger for example is classified as endangered. In 2022, the IUCN identified 28\% of 150 thousand evaluated species to be threatened with extinction. 

\begin{figure}[H]
	\centering
	\includegraphics[width=0.6\linewidth]{../images/IUCN}
	\caption{The different classes of the IUCN Red List}
	\label{fig:iucn}
\end{figure}
\ \\
The IBPES, the Intergovernmental Science-Policy Platform on Biodiversity, concluded that 1 million of species on Earth are threatened with extinction.

\subsubsection{How to measure biodiversity loss?}

Measuring biodiversity loss is complex and requires

\begin{itemize}
	\item evidence of a clear \textbf{causal relationship }between intervention and impact: you have to be sure that the change in biodiversity is caused by the human intervention you are studying
	\item the \textbf{absence of other confounding factors}: you have to be sure that the human intervention is the only thing influencing the biodiversity
	\item and clear decisions on \textbf{which elements of biodiversity are considered}: biodiversity has many different components, so it’s important to define which components will be measured and how
\end{itemize}
 \ \\
 It requires simplified proxies like indicator species, which are known to be sensitive for certain human-induced disturbances,  or multi-taxa approaches, to sense the impact on different groups of organisms. Various techniques are used to measure different dimensions of biodiversity, including satellites, eDNA, camera traps, and citizen science.\\
 \\
 The IUCN Red List and the Living Planet Index are major indicator tool for global biodiversity loss, as seen in section \ref{sec:recent-diversity-loss}.
\\
\\
Mid-point indicators?
\end{document}